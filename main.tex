\documentclass{article}

\title{Beast: the audacious musings of an aspiring software craftsman}
\author{Benjamin Paige}
\date{\today}

\begin{document}

\maketitle

\tableofcontents

\section{Introduction}

\section{The Notorious O. O. P.}
ABSTRACT CLASSES
For the most part, abstract classes should be used to encapsulate a core and common composition among objects within a package, but it should be kept secret.
Thus in java, I would keep it "package private".

Otherwise, public abstract classes should request implementation of some behavior by the client.
Much like working with an independent agent or broker, that serves as an intermediary for you and some service.
The abstract classes can generally be created anonymously. 

\section{Ironbrain: Tony Starks cheatsheets}

\section{Thoroughbred deliverables}
Never publish major releases with TODO tags in code
Never publish minor releases with FIXME tags in code

Write Javadocs incrementally, while maintaining a FIXME tag in each incomplete document until finished. 
Thus software will not be released with incomplete javadocs.
\end{document}
