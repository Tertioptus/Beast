\documentclass{article}

\title{Beast: the audacious musings of an aspiring software craftsman}
\author{Benjamin Paige}
\date{\today}

\begin{document}

\maketitle

\tableofcontents

\section{Introduction}

\section{The Notorious O. O. P.}
ABSTRACT CLASSES
For the most part, abstract classes should be used to encapsulate a core and common composition among objects within a package, but it should be kept secret.
Thus in java, I would keep it "package private".

Otherwise, public abstract classes should request implementation of some behavior by the client.
Much like working with an independent agent or broker, that serves as an intermediary for you and some service.
The abstract classes can generally be created anonymously. 

Static methods inextricably bind foreign code within the behavior of an object. This is a no no.  At least place the static method, like 'new', in a secondary constructor such that it can be overloaded.

\section{Advanced modality: Ironbrain: Tony Starks cheatsheets}

\section{Thoroughbred deliverables}
Never publish major releases with TODO tags in code
Never publish minor releases with FIXME tags in code

Write Javadocs incrementally, while maintaining a FIXME tag in each incomplete document until finished. 
Thus software will not be released with incomplete javadocs.

\section{The art of Storytelling}
Writing good software is more akin to authoring the most engaging literary prose. Every aspect of the end products should be sharing an experience with the user.
documentation surround an object's interface is more like a legend.  It's what the clients think about the object, and it all they care about.  What the object does in it's contruction and implementation, is what the object says about it self.  But know one cares about what you say about yourself, because it maybe smoking mirrors.  Thus the legend if validated through an array of testing
PRACTICALS
Java doc header should start with a small explanation of the chosen name for the class.  Then examples of what clients might experience with said class.Following each example with a small demonstration of how that experience may play out.

/*
 * [Explain naming choice]
 *
 * [Legend]
 *		[Examples {@link} ... ]
 *			[Demonstrations {@code} ... ]
 *
 * [References @see...]
 * ...
 */

Words not used:
imperative, declarative, functional, framework, toolkit, library, 
idiomatic, testing:proofs,assertions,affirmations,confirmations

Concepts not broached:
side effects, fault-tolerance

\end{document}
